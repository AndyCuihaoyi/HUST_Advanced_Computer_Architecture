\section{部分实验图表分析}
表格\ref{tab:scheme_features}系统梳理 AutoNUMA、TPP、Memtis 等 6 种主流方案的核心设计(如实现层级、跟踪策略、元数据管理方式),明确各方案的优势(如 HybridTier 轻量低耗、支持多页模式)与短板(如 Memtis 动态热度适配慢、元数据开销高),让不同方案的技术差异直观可辨,为分层系统选型提供直接参考。
\begin{table}[htbp]
  \centering
  \caption{6种主流内存分层方案核心特点}
  \label{tab:scheme_features}
  \resizebox{0.95\textwidth}{!}{% 适度加宽表格宽度
  \small  
    \begin{tabular}{@{}>{\bfseries}c p{10cm}@{}} % 序号列加粗,特点列设固定宽度自动换行
      \toprule
      序号 & \textbf{方案名称及核心特点} \\
      \midrule
      1 & \textbf{AutoNUMA}:通过LRU策略跟踪页面近期访问性,无频率维度识别能力,易导致冷热数据误分类。 \\
      \midrule
      2 & \textbf{TPP}:聚焦CXL内存的页级迁移优化,基于访问延迟反馈调整页面位置;依赖硬件NUMA节点模拟CXL特性,迁移决策仅考虑局部节点;因内核态实现,迁移开销略高。 \\
      \midrule
      3 & \textbf{Memtis}:支持持久化内存与CXL内存双层架构,通过频率直方图跟踪页面访问频率;采用哈希表存储精确计数,元数据开销高;对动态热度分布的适应速度较慢(分钟级)。 \\
      \midrule
      4 & \textbf{ARC}:轻量级混合LRU列表缓存算法,可区分单次与多次访问项;因不维护频率计数,无法区分热数据与温数据;在快层占比低的资源受限场景中性能衰减明显。 \\
      \midrule
      5 & \textbf{TwoQ}:采用FIFO+LRU双队列缓存策略,分离新访问与频繁访问页面;实现简单且缓存开销低,但对高倾斜度的访问分布适配性较差。 \\
      \midrule
      6 & \textbf{HybridTier}:通过频率+动量双指标跟踪适配动态热度;采用计数布隆过滤器(CBF)降低元数据内存与缓存开销;支持4KB页与2MB巨页。 \\
      \bottomrule
    \end{tabular}
  }
  \vspace{1pt}
  \footnotesize 注:所有方案均为应用透明型,无需修改业务代码即可部署。
\end{table}

表格\ref{tab:workload_features}提炼 CacheLib(CDN 缓存)、GAP(图计算)、Silo(数据库)等 6 类典型大内存负载的核心访问特征(如热点动态性、局部性、读写比例),覆盖互联网缓存、科学计算、AI 训练、数据库等关键应用场景,确保后续实验负载能代表真实业务需求,避免实验结论局限于单一场景。

\begin{table}[htbp]
  \centering
  \caption{6类测试负载核心特点}
  \label{tab:workload_features}
  \resizebox{0.95\textwidth}{!}{% 适度加宽表格宽度
  \small  
    \begin{tabular}{@{}>{\bfseries}c p{10cm}@{}} % 
      \toprule
      序号 & \textbf{负载类型及核心特点} \\
      \midrule
      1 & \textbf{CacheLib}:CDN缓存,热点动态变化 \\
      \midrule
      2 & \textbf{GAP}:图计算,随机访问局部性差 \\
      \midrule
      3 & \textbf{SPEC CPU 2017}:访问模式相对稳定 \\
      \midrule
      4 & \textbf{Silo}:数据库,热点集中读写混合 \\
      \midrule
      5 & \textbf{XGBoost}:机器学习,阶段性热点显著 \\
      \midrule
      6 & \textbf{BFS}:图遍历,热点随层级变化 \\
      \bottomrule
    \end{tabular}
  }
  \vspace{1pt}
\end{table}

\subsection{CacheLib工作负载实验分析}
\begin{figure}[htbp]
  \centering
  \includegraphics[width=0.95\textwidth]{images/cachelib_workload_analysis.png}
  \caption{CacheLib工作负载实验结果分析}
  \label{fig:cachelib_workload_analysis}
\end{figure}
\begin{enumerate}[label=\textbf{\alph*.}, itemsep=10pt]
    \item \textbf{实验目标(Experiment Target)} \\
    评估不同内存分层方案在CacheLib两类子工作负载(CDN、社交图谱)下的性能(延迟、吞吐量),验证各方案对不同快/慢内存比例的适配能力。

    \item \textbf{实验设计思想(Experiment Methodology)} \\
    选取6种内存分层方案(TPP、AutoNUMA、Memtis、ARC、TwoQ、HybridTier),在CacheLib的CDN与社交图谱子工作负载下开展测试;以“快层内存(DRAM):慢层内存(CXL内存)”的比例(1:16、1:8、1:4)为变量,分别测量各方案的中位数延迟与吞吐量指标。

    \item \textbf{实验具体配置(Experiment Configuration)} \\
    - 测试工作负载:CacheLib CDN工作负载、CacheLib社交图谱工作负载; \\
    - 内存分层方案:TPP、AutoNUMA、Memtis、ARC、TwoQ、HybridTier; \\
    - 内存比例:快层内存:慢层内存 = 1:16、1:8、1:4; \\
    - 测量指标:中位数延迟(单位:微秒)、吞吐量(单位:Mop/s,每秒百万操作数)。

    \item \textbf{图表标记含义(x-axis and y-axis)} \\
    - 横坐标:快层内存与慢层内存的比例(1:16、1:8、1:4); \\
    - 纵坐标:
      \begin{itemize}[label=-]
        \item 左图:中位数延迟(单位:$\mu$s,微秒);
        \item 右图:吞吐量(单位:Mop/s,每秒百万操作数);
      \end{itemize}
    - 图表类型:分组柱状图; \\
    - 系列标记:不同颜色代表不同内存分层方案(TPP:灰色,AutoNUMA:橙色,Memtis:黄色,ARC:深蓝色,TwoQ:绿色,HybridTier:浅蓝色)。

    \item \textbf{实验结论(Experiment Conclusion)} \\
    在CacheLib两类工作负载下,HybridTier方案的综合性能(低延迟、高吞吐量)显著优于其他5种方案;Memtis的性能波动较大(在1:4比例下延迟明显升高),而HybridTier在不同内存比例下的性能表现更稳定。

    \item \textbf{结果解释(Experiment Explanation)} \\
    - HybridTier采用“频率+动量”双指标跟踪,能更好适配CacheLib工作负载的动态热点变化,因此在不同内存比例下均保持低延迟与高吞吐量; \\
    - Memtis依赖频率直方图跟踪热点,对CacheLib的动态热点适应速度慢,当快层内存占比降低(如1:4)时,易出现冷页误晋升导致的快层内存污染,进而推高延迟、降低吞吐量; \\
    - 其他方案(如AutoNUMA、ARC)因仅跟踪单一维度(近期访问性),无法精准识别动态热点,性能表现弱于HybridTier。
\end{enumerate}


\subsection{动态访问分布适应时间实验分析}
\begin{figure}[htbp]
  \centering
  \includegraphics[width=0.95\textwidth]{images/dynamic_access_distribution_adaptation_time.png}
  \caption{动态访问分布适应时间实验结果分析}
  \label{fig:dynamic_access_distribution_adaptation_time}
\end{figure}
\begin{enumerate}[label=\textbf{\alph*.}, itemsep=10pt]
    \item \textbf{实验目标(Experiment Target)} \\
    评估不同内存分层方案在访问分布发生变化后,达到稳态性能所需的适应时间,验证各方案对动态访问模式的响应能力。

    \item \textbf{实验设计思想(Experiment Methodology)} \\
    选取Memtis与HybridTier两种内存分层方案,在CDN和Social-graph两类工作负载下,测试不同快/慢内存比例(1:16、1:8、1:4)下,方案延迟达到稳态中位数延迟1%以内所需的时间;通过对比两类方案的适应时间,量化HybridTier的动态适配优势。

    \item \textbf{实验具体配置(Experiment Configuration)} \\
    - 测试工作负载:CDN工作负载、Social-graph工作负载; \\
    - 内存分层方案:Memtis、HybridTier; \\
    - 内存比例:快层内存:慢层内存 = 1:16、1:8、1:4; \\
    - 评估指标:适应时间(单位:分钟,定义为延迟接近稳态的时间)。

    \item \textbf{表格标记含义} \\
    - 行维度:包含2种内存分层方案(Memtis、HybridTier)及两者的适应时间相对减少幅度; \\
    - 列维度:分为CDN与Social-graph两类工作负载,每类负载下包含3种快/慢内存比例(1:16、1:8、1:4); \\
    - 数据项:单元格数值为对应方案在特定负载与内存比例下的适应时间(单位:分钟),“>60”表示适应时间超过60分钟。

    \item \textbf{实验结论(Experiment Conclusion)} \\
    HybridTier对动态访问分布的适应速度显著优于Memtis:在两类工作负载与各内存比例下,HybridTier的适应时间均远短于Memtis,且适应时间的稳定性更强;相对Memtis,HybridTier的适应时间最多可减少5.9倍。

    \item \textbf{结果解释(Experiment Explanation)} \\
    - Memtis依赖频率直方图跟踪访问模式,更新频率低、对动态变化响应慢,因此在多数场景下适应时间超过60分钟; \\
    - HybridTier采用“频率+动量”双指标跟踪,动量指标可快速捕捉短期访问变化,因此能在25分钟内(CDN负载)或10分钟内(Social-graph负载)适应新的访问分布; \\
    - 内存比例对Memtis的适应时间影响极大(如Social-graph负载下1:8比例需>60分钟、1:4比例仅需29.6分钟),而HybridTier的适应时间受内存比例影响较小,体现了其更强的场景适配稳定性。
\end{enumerate}

