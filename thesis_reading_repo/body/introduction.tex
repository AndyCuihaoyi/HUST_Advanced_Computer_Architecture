\section{论文写作大纲}

\subsection{摘要}
现代工作负载对内存容量的需求日益增长。基于Compute Express Link
(CXL)的内存分层技术已成为解决此问题的有前景的方案,其通过将传统
DRAM与慢速层级的CXL内存设备结合使用。通过分析先前的分层系统,
观察到高性能内存分层面临的两大挑战:在适应倾斜但动态变化的数据热度分布的同时,将分层导致的内存和缓存开销最小化。
为应对这些挑战,我们提出了HybridTier——适用于CXL内存的
自适应轻量级分层系统。HybridTier同时追踪数据的长期访问频率与
短期访问动量,以准确捕捉并适应不断变化的热度分布。HybridTier
通过概率性地追踪数据访问来降低元数据内存开销,以较小的追踪精度损失
换取更高的内存效率,而该精度损失对应用性能的影响可忽略不计。为减少
缓存开销,HybridTier采用优化数据局部性的轻量级数据结构来追踪数据热
度。我们的评估表明,HybridTier的性能比以往系统提升高达91\%(几何平
均值19\%),内存开销降低2.0--7.8倍,缓存未命中降低1.7--3.5倍。
\subsection{引言}
现代应用对内存容量和带宽的需求日益增加。单个服务器内存容量和内存成本都成为限制,近十年DRAM密度的发展也相对缓慢。以CXL为基础的分层内存架构是解决上述问题的可靠方案。更大容量更低价格的CXL内存与本地DRAM相结合,能耗更低使用效率更高。另一方面,CXL内存的延迟和带宽表现不如本地DRAM,所以要将高速层级的DRAM和低速层级的CXL Memory巧妙结合。

然而,实现高性能分层颇具挑战。我们对大内存工作负载进行了分析,并得出两个观察结果:(1)实际工作负载通常表现出倾斜但动态变化的数据热度分布,如图\ref{fig:data_changes}所示;(2)管理数据访问统计信息可能带来显著开销。理想的分层系统应满足三个要求:(1)通过将最热数据放置在快速层级内存中,准确捕获热数据集合;(2)快速适应热度分布的变化;(3)最小化分层元数据开销。

\begin{figure}[htbp]  % 尝试将浮动元素放在页面合适位置
\centering
\includegraphics[width=0.8\textwidth]{images/data_changes.png}
\caption{负载热度分布变化}
\label{fig:data_changes}
\end{figure}

然而,现有系统无法满足全部三个要求。一类工作采用基于频率的分层,通过存储每个页面的访问计数来构建热度直方图。基于此直方图,分层系统将最热页面置于快速层级,满足要求1。另一类工作是基于近期性的分层,它使用访问新近性来近似数据热度。此类系统使用连续缺页之间的时间等指标来做出数据放置决策。基于近期性的系统满足要求3,因为与频率对应项相比开销更小。然而,基于近期性的系统无法满足要求1也不满足要求2。

在本工作中,提出HybridTier,一种应用透明的分层系统,它满足了这三个要求。HybridTier为每个页面维护两个指标,以同时捕获长期访问历史和短期热度变化。使用计数布隆过滤器(CBF)追踪内存访问,这是一种概率性数据结构,以较高的内存效率换取较低的追踪准确性。

本文做出以下贡献:
\begin{itemize}
\item 分析了现有分层系统,揭示了三个新发现:1)适应变化的热度具有挑战性,导致在真实工作负载下性能次优;2)维护历史访问信息可能带来高元数据内存开销;3)分层系统可能因元数据更新期间数据局部性差而遭受大量缓存未命中;
\item 介绍HybridTier,一种应用透明的内存分层系统,具有自适应性和轻量级特性。采用新颖的访问追踪方法,同时捕获长期热度分布和短期数据热度变化。同时,通过采用概率性访问追踪和局部性优化的数据结构,显著降低了元数据内存消耗和缓存未命中;
\item 比较了HybridTier在六种大内存工作负载上的性能,同时改变快速到慢速层级内存比例。HybridTier的性能平均优于先前工作19\%,同时因分层产生的内存开销降低2.0−-7.8倍,缓存未命中降低1.7--3.5倍。
\end{itemize}

\subsection{背景和动机}

\subsection{混合层级关键思想}

\subsection{混合层级设计}

\subsection{实验设计}

\subsection{系统评估}

\subsubsection{研究背景与趋势}
\begin{enumerate}
    \item 参考文献必须全部在论文正文中按顺序引用
    \item 不得在标题处进行引用,引用出现在标点之前。
    \item 可以在引用处右上角加标注进行引用,也可以直接在正文中用“文献[3]指出……”这样的话语进行引用
    \item 参考文献在后面的【参考文献】中排列顺序按照在论文中第一次引用的先后顺序排列
    \item 多篇参考文献群引不超过3篇,可使用如下风格:[1,2,3]、[1][2][3]
\end{enumerate}

\subsubsection{面临的问题和挑战}

\subsubsection{课题目的与意义}

阐述最终的输出成果能够达到什么目标,比如什么样的性能或者提供什么样的功能之类。

\subsection{国内外研究现状}
通过大量文献阅读,对所研究内容进行综述,较为详细的说明研究内容的国内外现状,
建议按照时间轴分阶段说明或者按照原理/机制分类别说明,不论哪种方式,
都要对每个阶段/每个类别的工作原理、机制、试图解决的问题、解决了的问题和存在的不足进行阐述和分析,
该部分篇幅在3-4页为宜

一次引用建议不要超过三篇文献,文献引用按照:递增顺序、全部引用的原则进行引用

\subsection{论文的主要内容与结构}
\subsubsection{论文的主要内容}

\subsubsection{论文结构}
本文的主要内容如下:

第一章...

第二章...

\dots

第六章...


% 插入图片,如图\ref{fig:fig1}所示:

% \cfig{fig1}{0.5}{插入图片示例}

% 公式,如公式\ref{eq:1}所示:
% \begin{equation}
%     \label{eq:1}
%     f_2 = f_v + f_a + f_{\omega}
% \end{equation}

% 列表,如\ref{chart:1}所示:
% \begin{table}[!ht]
%     \centering
%     \caption{歪比巴伯}
%     \label{chart:1}
%     \begin{tabular}{ccc}
%     \toprule
%         A & B & C  \\ \midrule
%         I & 1 & 2  \\ 
%         II & 3 & 4 \\ \bottomrule
%     \end{tabular}
% \end{table}

% 还可以添加算法\ref{al:1}:
% \begin{algorithm}[ht]
%     \caption{123}\label{al:1}
%     \begin{algorithmic}[1]
%         \Require a
% 			\Ensure b
%             \State v
% 			\While{q}
%             \State 1
% 			\If {d}
%             \State 3
% 			\State \Return dd
%             \Else
%             \State 2
%             \EndIf	
% 			\EndWhile
%     \end{algorithmic}
% \end{algorithm}
